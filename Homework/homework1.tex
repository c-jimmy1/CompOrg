\documentclass[12pt]{article}

\usepackage[shortlabels]{enumitem}
\usepackage[margin=1in]{geometry}		% For setting margins
\usepackage{amsmath}				% For Math
\usepackage{fancyhdr}				% For fancy header/footer
\usepackage{graphicx}				% For including figure/image
\usepackage{cancel}					% To use the slash to cancel out stuff in work

%%%%%%%%%%%%%%%%%%%%%%
% Set up fancy header/footer
\pagestyle{fancy}
\fancyhead[LO,L]{Jimmy Chen}
\fancyhead[CO,C]{CSCI 2500 - Computer Organization}
\fancyhead[RO,R]{October 2, 2023s}
\fancyfoot[LO,L]{}
\fancyfoot[CO,C]{\thepage}
\fancyfoot[RO,R]{}
\renewcommand{\headrulewidth}{0.4pt}
\renewcommand{\footrulewidth}{0.4pt}
%%%%%%%%%%%%%%%%%%%%%%
\begin{document}
\section{Homework 1}
\subsection{Textbook Problem 4.15.5}
Consider three different processors P1, P2, and P3 executing the same instruction set. P1 has a 3 GHz clock rate and a CPI of 1.5. P2 has a 2.5 GHz clock rate and a CPI of 1.0. P3 has a 4.0 GHz clock rate and has a CPI of 2.2.

\begin{enumerate}[(a)]
\item Which processor has the highest performance expressed in instructions per second?
Answer:
\begin{center}

\end{center}
\item If the processors each execute a program in 10 seconds, find the number of cycles and the number of instructions.
\item We are trying to reduce the execution time by 30\% but this leads to an increase of 20\% in the CPI. What clock rate should we have to get this time reduction?
\end{enumerate}




\subsection{Textbook Problem 4.15.8}
\subsection{Textbook Problem 4.15.10}
\subsection{Textbook Problem 4.15.13}
\subsection{Textbook Problem 4.15.14}
\subsection{Textbook Problem 4.15.15}
\subsection{Textbook Problem 4.15.16}


\end{document}


